\documentclass[10pt,a4paper]{scrartcl}

\DeclareOldFontCommand{\rm}{\normalfont\rmfamily}{\mathrm}
\DeclareOldFontCommand{\sf}{\normalfont\sffamily}{\mathsf}
\DeclareOldFontCommand{\tt}{\normalfont\ttfamily}{\mathtt}
\DeclareOldFontCommand{\bf}{\normalfont\bfseries}{\mathbf}
\DeclareOldFontCommand{\it}{\normalfont\itshape}{\mathit}
\DeclareOldFontCommand{\sl}{\normalfont\slshape}{\@nomath\sl}
\DeclareOldFontCommand{\sc}{\normalfont\scshape}{\@nomath\sc}

\usepackage{isabelle,isabellesym}

\usepackage{color}
\definecolor{linkcolor}{rgb}{0,0,0.7}
\usepackage[colorlinks=true,linkcolor=linkcolor,citecolor=linkcolor,
            filecolor=linkcolor,urlcolor=linkcolor]{hyperref}

\hypersetup{
pdfauthor={Matthew Brecknell},
pdfsubject={A puzzle about parities and permutations},
pdftitle={Schr\"odinger's hats}
}

\parindent 0pt\parskip 0.5ex

\setlength{\oddsidemargin}{0cm}
\setlength{\evensidemargin}{0cm}
\setlength{\topmargin}{-1cm}
\setlength{\textwidth}{15.5cm}
\setlength{\textheight}{22.5cm}

\newcommand{\meth}[1]{\texttt{#1()}}
\newcommand{\obj}[1]{\textsf{\small #1}}

\begin{document}

\title{Schr\"odinger's hats}
\subtitle{A puzzle about parities and permutations}
\author{Matthew Brecknell}

\maketitle

Meet Schr\"odinger, who travels the world with an unusually clever clowder of
$n$ talking cats. In their latest show, the cats stand in a line.
Schr\"odinger asks a volunteer (not a plant!) to take $n+1$ hats, numbered zero
to $n$, and randomly assign one to each cat, so that there is one spare.  Each
cat sees all of the hats in front of it, but not its own hat, nor those behind,
nor the spare hat. The cats then take turns, each calling out a single number
from the set $\{0..n\}$, without repeating any number previously called, and
without any other communication. Although the first call is allowed to be
wrong, the remaining cats always call out the numbers on their own hats.

\input{session}

\end{document}
